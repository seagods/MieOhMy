\documentstyle[12pt]{article}
\setlength{\textheight}{9.80in}
\setlength{\textwidth}{6.40in}
\setlength{\oddsidemargin}{0.0mm}
\setlength{\evensidemargin}{1.0mm}
\setlength{\topmargin}{-0.6in}
\setlength{\parindent}{0.2in}
\setlength{\parskip}{1.5ex}
\newtheorem{defn}{Definition}
\renewcommand{\baselinestretch}{1.2}

\begin{document}

\bibliographystyle{prsty}


\thispagestyle{empty}

\title{Electromagnetic Scattering from a Sphere}
\author{Chris Godsalve}
\maketitle

\tableofcontents

\section{Foreword}

My purpose in writing this article is to write up some of my own notes. 
Why publish it on the Net? Of course, I could lose my notes by some 
accident, and then I should not be able to refer to them. However, if some
reader happens upon them, and is interested in the subject, it may (or may not)
 be useful to them. At any rate, though there are many standard texts, looking at the problem through yet another set of eyes should do no harm. This is very much a "work in progress". I shall come back to them, make additions, correct errors (and there may be many of them) and so on over time.

In the afterword, the reader is directed to a National Centre for Atmospheric Research report which may be downloaded, and this in turn provides a link so that the reader may download some extremely well written computer code that performs the Mie theory calculations.

\section{Introduction}

The subject of electromagnetic scattering from a sphere has a long history. The completely general solution was beyond some of the best known 19th century mathematicians and physicists involved in the field. It was not until 1908 that Gustav  Mie solved the problem for electromagnetic  scattering from metal spheres, and Ludwig Lorenz, and Peter Debye managed a completely general solution. The solution for the theory of electromagnetic radiation from a sphere is now usually called the Mie theory, or sometimes the Mie-Lorenz theory. These days, the subject is of much interest to anyone working in the field of atmospheric radiation, but the theory is also important in other fields of research.

Of course, the basis of the study lies in Maxwell's equations. However, we are still greatly indebted to Heaviside for casting Maxwell's equations in vector form. The basis of the treatment given here is based on Liou's book on atmospheric radiation \cite{Liou:Mybib}. However, there are alterations, additions, and changes in the order of argument given here. Also, further material is included from  a book by Deirmendjian \cite{Deirmendjian:Mybib}. Reprints of the latter are available from {\it RAND}. Just Google on RAND reprints and Deirmendjian to order a copy if required.

\section{Maxell's Equations}

We shall simply state Maxwell's equations, which  are

\begin{equation}
 \nabla \cdot {\bf E}=\frac{\rho}{\epsilon_0},
\end{equation}

\begin{equation}
\nabla \times {\bf E}=-\frac{\partial {\bf B}}{\partial t},
\end{equation}

\begin{equation}
 \nabla \cdot {\bf B}=0,
\end{equation}

\begin{equation}
c^2 \nabla \times {\bf B}=\frac{\partial {\bf E}}{\partial t} +\frac{{\bf J}}{\epsilon_0}.
\end{equation}
The vectors ${\bf E }$ and ${\bf B}$ are the electric field and the magnetic flux density, the scalar $\rho$ is the charge density, and the vector ${\bf J}$
is the current density. In these equations $\epsilon_0$ is the permeability of free space, and has the value of 8.8542 Farads per metre. The magnetic permeability of space ($\mu_0)$ is defined to be $4 \pi \times 10^{-7}$ Webers per Ampere metre, so that the speed of light is given by
\begin{equation}
c =\frac{1}{ \mu_0 \epsilon_0}.
\end{equation}

In any material, the current density can be split up into different parts.
There is the "true" current, due to electrons drifting through the material.
Apart from that, charges bound to atoms will oscillate as any material is polarisable, and also the material may be magnetised, so there are circulating currents. To describe these two effects we introduce two new fields
 ${\bf P}$ and ${\bf M}$. These are the electric and magnetic dipole moment per unit volume. 
With these new fields we may write eqn.4 as follows.
\begin{equation}
c^2 \nabla \times {\bf B}=\frac{\partial {\bf E}}{\partial t} 
+\frac{
{\bf J}_e + \nabla \times {\bf M}+\frac{\partial {\bf P}}{\partial t}
}{\epsilon_0}.
\end{equation}
Here ${\bf J}_e$ is the current due to conduction. For a good discussion on this
see \textsection 36 of the Feynman Lectures on Physics (volume 2) \cite{FeynLect:Mybib}.

\begin{equation}
c^2 \nabla \times \left \lbrack
 {\bf B}- \frac{{\bf M}}{\epsilon_0 c^2}
 \right \rbrack
=\frac{\partial }{\partial t} \left  \lbrack {\bf E}+ \frac{{\bf P}}{\epsilon_0} \right \rbrack 
+\frac{J_e}{\epsilon_0}.
\end{equation}
This is in the form of Maxwell's equations still, and so the 
displacement field ${\bf D}$ is introduced
\begin{equation}
{\bf D}= \epsilon_0 E+{\bf P},
\end{equation}
and the magnetic or auxiliary field ${\bf H}$ which is
\begin{equation}
{\bf H}={\bf B}-\frac{{\bf M}}{\epsilon_0 c^2}.
\end{equation}
Then we have
\begin{equation}
 \nabla \cdot {\bf D}=\frac{\rho}{\epsilon_0},
\end{equation}
\begin{equation}
\nabla \times {\bf E}=-\frac{\partial {\bf B}}{\partial t},
\end{equation}
\begin{equation}
 \nabla \cdot {\bf B}=0,
\end{equation}
\begin{equation}
\epsilon_0 c^2 \nabla \times {\bf H}=\frac{\partial {\bf D}}{\partial t} 
+{\bf J}.
\end{equation}
We add that the Lorenz force law is
\begin{equation}
{\bf F}=q ( {\bf} E + {\bf v} \times {\bf B}),
\end{equation}
and that the magnitude of the force between two charges is
\begin{equation}
\vert {\bf F} \vert= \frac{ q_1 q_2}{ 4 \pi \epsilon_0 r^2}
\end{equation}

\subsection{Units!}

The equations above are all in MKS SI units. However there are also 
Gaussian units which are in CGS, and Heaviside-Lorenz units. Physics undergraduate courses generally teach everything in SI units, but much work done in
electromagnetism is done using CGS. This is most annoying for the graduate student, but there are good reasons for the adherence to Gaussian units, but the changeover requires care so we shall make a brief aside to discuss the units that are used and how Maxwell's equations are affected by them.

First off, there are two fundamental experiments. One is the force between two
charges, and the other is the force between two wires in which currents are flowing. Now, we have to decide what one unit of charge is, and depending on our choice of what a unit charge is, there will be a proportionality constant. So
the force between two charges as given by Coulomb's law is
\begin{equation}
F=k_e \frac{q_1 q_2}{r^2}.
\end{equation}
Of course, the converse is true. We can decide what $k_e$ is arbitrarily, and that will define what a unit charge is. This is the way things are done.
Also, we have the magnitude of the  force between two wires of length L carrying currents $i_1$ and $i_2$. That is Ampere's law.
\begin{equation}
F=k_m \frac{ i_1 i_2 L}{r}
\end{equation}
Now, you can decide what value for $k_m$ or $k_e$, but you must take care. They are not independent. That is because the charge flow through a wire is the current multiplied by the time.  This means that $k_e$ and $k_m$ must be related. 
If you decide on one, you have also decided the other. The ratio of the two
constants is some constant.
If you divide one equation by the other you see that
\begin{equation}
  \frac{k_e}{k_m} \frac{ [Q]^2}{[L]^2} \times \frac{[T]^2}{[Q]^2} \> {\rm is \> dimensionless}.
\end{equation}
That is to say that the ratio of the two is some universal constant which has the dimensions of the square of a velocity. In fact it turns out that
\begin{equation}
\frac{k_e}{k_m}=\frac{c^2}{2}.
\end{equation}

But to confuse matters further, CGS has {\it two} ways of defining things.
 One way to proceed  is the basis of {\it electrostatic units} or e.s.u. in which $k_e=1$
 which in turn fixes $k_m$. There are also 
{\it electromagnetic} units, or e.m.u. In this case $k_m=2$, which in turn fixes
$ k_e$.

In e.s.u, the unit of charge is the statcoulomb, that is $0.1/c$ coulombs.
(We shall discuss SI units later, here we shall just mention the conversion factors.) One statcoulomb is then 3.33564095$\times 10^{-10}$  coulombs, and one coulomb is
2.99792458$ \times 10^9$ coulombs. The large differences are due to the ratio 
of $k_e$ and $k_m$ being related to the speed of light, and different choices
made as to what they should be. Currents are measured in statamperes, so
1 statampere =10$c$ amperes. (Charge flow is current times time, and the
the time units in seconds for both SI and CGS.) The statampere is sometimes 
just called an e.s.u of current. The prefix {\it stat} in front of any quantity tells you it is
in $e.s.u.$. If the prefix is {\it ab} then the units are in e.m.u..

Now, how can we compare SI and CGS for e.m.u. We have {\it two dyne} ($10^{-5}N$)
 of force per centimetre of wire for one e.m.u of current in two wires 1cm apart. In SI $k_m=2 \times 10^{-7}$ by definition. So there is a  force of 1N 
 per metre of length  between two wires 1m apart each carrying one ampere.
That is
\begin{equation}
(1 e.m.u)^2 = \frac{10^{-5} N}{2}
$$ and   $$
(1 A)^2 = \frac{1 N}{2 \times 10^{-7}}
\end{equation}
So, 1 e.m.u (or {\it abampere}) of current is 10A. and one {\it abcoulomb}
is 10 coulombs. Comparison of the abampere and statampere gives 1 abampere =$c$ statamperes.

Confused? You soon will be. Now we come to Gaussian units which take magnetic
quantities from e.m.u. and electrostatic quantities from e.s.u. These are the most generally used units in the CGS system.
 First off
there is the gauss. The force in dynes per centimetre of a wire
is the magnetic flux density $B$ in gauss times the current in abamperes. The quantity of 1 gauss is $10^{-4}$ Tesla. The unit for magnetic field strength is the Oersted, this is defined as equal to one at the centre of of a circular current carrying $1/2 \pi$ amperes of current, the radius of the loop being one centimetre.
One Oersted is $250/\pi$ amps  per metre. Note there if the loop is surrounding a vacuum, if you divide the flux density in gauss by the magnetic field 
strength in oersteds you get 1. There is no need for an absolute permeability.
If the loop surrounds an iron core, then you may have something like 8000 oersteds. In this system, the fundamental units for the magnetic flux density and the magnetic field strength are the same.

Let's take a look at SI units again. Here we have
\begin{equation}
\mu_0=2 \pi k_m,  \> \> \epsilon_0=\frac{1}{4 \pi k_e}
$$ so that  $$
c^2=\frac{1}{\mu_0 \epsilon_0}.
\end{equation}
(Recall that $k_m=2 \times 10^{-7}$ for SI units.) Now, in CGS units the constants are dimensionless. All quantities are derivable from the dimensions of mass, length, and time. In fact, SI could have made the definition of current in this way, however SI treats the ampere as a fundamental unit like the second or the metre. So $k_m$ in SI is not dimensionless but has units of newtons per ampere squared.

Let's do a direct conversion of the magnitude of the force in dynes, to the magnitude of the force in newtons.
\begin{equation}
\frac{q_1 q_2}{ r^2 }=F_d,
\end{equation}
where the subscript $d$ reminds us that the force is in dynes.
If we change $r$ cm to $R$ m, and use the fact that 1 statcoulomb
is $0.1/c$ coulombs, then we have
\begin{equation}
F_d= 
  \frac {  q_1/10 c \times 10 c \times q_2/10 c \times 10 c }
  {   (r / 100)^2 \times {100^2} }
$$   $$
  =\frac {  Q_1 Q_2 100 c^2 }
  { 10^4 R^2}
\end{equation}
where the $Q$s are the charges in Coulombs, and $R$ is the distance in metres.
That is 
\begin{equation}
F_d= 
  \frac {  Q_1 Q_2  }
  { 100 \mu_0 \epsilon_0 R^2 }
\end{equation}
 Now if we {\it define}
$\mu_0= 4 \pi \times 10^{-7}$ this is
\begin{equation}
F_d = \frac{ Q_1 Q_2}{ 4  \pi \times 10^{-5} \epsilon_0 R^2}
   = 10^{5} \frac{ Q_1 Q_2}{ 4  \pi \epsilon_0 R^2},
$$ or  $$ 
10^{-5} F_d  
   =  \frac{ Q_1 Q_2}{ 4  \pi \epsilon_0 R^2}=F_N,
\end{equation}
So, if $F_d$ were 1 dyne, then the force would be $10^{-5}$ newtons.




In Gaussian  units we, we pick up factors of $4 \pi$, but we also 
pick up factors of $1/c$. Maxwell's equations in this system  read
\begin{equation}
 \nabla \cdot {\bf D}= 4 \pi {\rho}
\end{equation}
\begin{equation}
\nabla \times {\bf E}=- \frac{1}{c} \frac{\partial {\bf B}}{\partial t},
\end{equation}
\begin{equation}
 \nabla \cdot {\bf B}=0,
\end{equation}
\begin{equation}
\nabla \times {\bf H}= 
\frac{1}{c} \frac{\partial {\bf D}}{\partial t} +
\frac{ 4 \pi}{c} {\bf J}
\end{equation}
The displacement field and the magnetic  field are then
\begin{equation}
{\bf D}={\bf E}+4 \pi {\bf P},
\end{equation}
and
\begin{equation}
{\bf H}={\bf B}-4 \pi {\bf M}.
\end{equation}
In Gaussian units the Lorenz force law is
\begin{equation}
{\bf F}=q ( {\bf E} + \frac{1}{c} {\bf v} \times {\bf B}).
\end{equation}

We should add there are scale changes that eliminate the factors of $4 \pi$, and these units are called
Lorentz-Heaviside units. If these are used, with natural units such that $c=1$, Maxwell's equations are of their simplest 
form.

\section{The Helmholtz Wave Equation}

In the following, we shall only consider isotropic media, and shall the form
of Maxell's equations as in equations 26 to 29.
Taking divergences of the eqn.29, we see that
\begin{equation}
\nabla \cdot {\bf J}+\frac{\partial \rho}{\partial t}=0.
\end{equation}
which is the equation of continuity. Since we are considering isotropic materials
we a scalar conductivity, permittivity and permeability so that 
${\bf J}=\sigma {\bf E}$,
${\bf D}=\epsilon  {\bf E}$,
${\bf B}=\mu {\bf H}$, and we assume the {\it net} charge density is zero.
Then,
\begin{equation}
 \nabla \cdot {\bf E}= 0, 
\end{equation}
\begin{equation}
\nabla \times {\bf E}=- \frac{\mu}{c} \frac{\partial {\bf H}}{\partial t},
\end{equation}
\begin{equation}
 \nabla \cdot {\bf H}=0,
\end{equation}
\begin{equation}
\nabla \times {\bf H}= 
\frac{\epsilon}{c} \frac{\partial {\bf E}}{\partial t}.
\end{equation}
Now we assume the fields to have the form 
${\bf E}={\bf E}({\bf r}) \> e^{i \omega t}$, and
${\bf H}={\bf H}({\bf r}) \> e^{i \omega t}$
Then we have
\begin{equation}
\nabla \times {\bf E}=- i k {\bf H},
\end{equation}
and
\begin{equation}
\nabla \times {\bf H}=- i k m^2 {\bf E}.
\end{equation}
Here, we have put $m=\sqrt{\epsilon}$, and $k=2 \pi / \lambda=\omega/c$.
Now we use the identity
\begin{equation}
\nabla \times (\nabla \times {\bf A})=\nabla (\nabla \cdot {\bf A} )
 -\nabla^2 {\bf A},
\end{equation}
where ${\bf A}$ is any arbitrary vector field (and not the vector potential).
From this we see that
\begin{equation}
\nabla^2 {\bf E}= -km^2 {\bf E}, 
$$   and $$
\nabla^2 {\bf H}= -km^2 {\bf H}. 
\end{equation}
That is, both fields obey the vector Helmholtz equation.

\section{The General Solution in Spherical coordinates}

Our goal is to solve the problem for the scattering of a plane electromagnetic
waves off a sphere. It is therefore  no surprise to see a 
spherical coordinate system introduced, as in Fig.1.
\vspace*{10cm}
\begin{figure}[htb]
\special{psfile=Fig1.eps vscale=50 hscale=50 voffset=10 hoffset=100}
\caption{The spherical coordinates as used in this article}
\end{figure}

Before we proceed, the author should make the reader aware, that as vectors are invariants, or tensors of rank one if you will, a vector equation is independent of the coordinate system used. So some of the following discussion may be skipped over. The author does consider that it is useful to review the vector Laplacian in spherical coordinates, and if the reader disagrees the reader may skip it!

\subsection{The Scalar Helmholtz Equation}

Though we are dealing with the Laplacian of a vector field (which makes life hard
 in spherical coordinates) we happen to know in advance that the solution of the
vector equation  requires the solution of the scalar Helmholtz equation. This is
now our starting point. The scalar Laplacian in spherical coordinates can be written as
\begin{equation}
\nabla^2=\frac{1}{r^2} \frac{\partial }{\partial r} \left ( r^2 \frac{\partial}{\partial r} \right )
+\frac{1}{ r^2 \sin \theta} \frac{\partial }{\partial \theta} \left ( \sin \theta \frac{\partial}{\partial \theta } \right )
+\frac{1}{r^2 \sin^2 \theta} \frac{\partial^2}{\partial \phi^2}.
\end{equation}

We shall now consider the scalar equation
\begin{equation}
\nabla^2 \psi +k^2 m^2 \psi=0.
\end{equation}
In time honoured fashion, we put $\psi= R(r) \Theta(\theta)\Phi(\phi)$ and separate variables.
Then one arrives at
\begin{equation}
\frac{1}{r^2} \frac{1}{R} \frac{\partial}{\partial r} \left ( r^2 \frac{\partial R}{\partial r} \right )
+ \frac{1}{r^2 \sin \theta} \frac{1}{\Theta} \frac{\partial}{\partial \theta}
\left ( \sin \theta \frac{\partial \Theta}{ \partial \theta} \right )
+\frac{1}{r^2 \sin^2 \theta} \frac{1}{\Phi} \frac{\partial^2 \Phi}{\partial \phi}+k^2 m^2=0.
\end{equation}
Then
\begin{equation}
\sin^2 \theta \frac{1}{R} \frac{\partial}{\partial r} \left ( r^2 \frac{\partial R}{\partial r} \right )
+  \sin \theta \frac{1}{\Theta} \frac{\partial}{\partial \theta}
\left ( \sin \theta \frac{\partial \Theta}{ \partial \theta} \right )
+ \frac{1}{\Phi} \frac{\partial^2 \Phi}{\partial \phi}+r^2 \sin^2 \theta \> k^2 m^2=0.
\end{equation}
Now, there is a term in $\phi$ alone, which must be matched to the terms in
the other variables, so we may write
\begin{equation}
\frac{1}{\Phi} \frac{ \partial^2 \Phi}{\partial \phi^2}=constant.
\end{equation}
Of course, $\Phi(\phi+2 \pi)=\Phi(\phi)$ so that 
\begin{equation}
\Phi(\phi)=a_l \cos( l \phi)+b_l \sin(l \phi),
\end{equation}
where $a_l$ and $b+l$ are arbitrary constants. So, we set the constant term of eqn.46 to be $-l^2$. Then we have
\begin{equation}
 \frac{1}{R} \frac{\partial}{\partial r} \left ( r^2 \frac{\partial R}{\partial r} \right )
+ \frac{1}{\sin \theta} \frac{1}{\Theta} \frac{\partial}{\partial \theta}
\left ( \sin \theta \frac{\partial \Theta}{ \partial \theta} \right )
-\frac{l^2}{ \sin^2 \theta} +k^2 m^2 r^2=0.
\end{equation}
Now we have terms in $\theta$ and $r$ alone. So we may now put
\begin{equation}
 \frac{1}{R} \frac{\partial}{\partial r} \left ( r^2 \frac{\partial R}{\partial r} \right )
 +k^2 m^2 r^2=const,
\end{equation}
and
\begin{equation}
 \frac{1}{\sin \theta} \frac{1}{\Theta} \frac{\partial \Theta}{\partial \theta}
-\frac{l^2}{ \sin^2 \theta} =-const.
\end{equation}
First, we shall consider eqn.50. On putting $\mu=\cos \theta$, this becomes
\begin{equation}
\frac{d}{ d \mu} (1-\mu^2) \frac{ d \Theta}{d \mu}
+ \left \lbrack const=\frac{l^2}{1 -\mu^2} \right \rbrack=0.
\end{equation}
The value of this constant is entirely arbitrary, and if we compare this with the associated Legendre differential equation \cite{AbramowitzStegun:Mybib}
\begin{equation}
\frac{d}{ d x} (1-x^2) \frac{ d y}{d x}
+ \left \lbrack m(m+1)-\frac{l^2}{1 -x^2} \right \rbrack=0.
\end{equation}
We choose our constant so to be $m(m+1)$ (remember that the complete solution will end up being linear combinations of our functions, so that choosing the constant merely scales the coefficients in the series expansion that we are going to end up with. So, we now have
\begin{equation}
\Theta(\theta)=P^m_l(\cos \theta),
\end{equation}
where the $P^m_L$ are the associated Legendre functions, which are well documented in \cite{AbramowitzStegun:Mybib}.

We now turn our attention to $R(r)$. If we put
\begin{equation}
R=\frac{1}{\sqrt{\rho}} Z(\rho), 
\end{equation}
where
\begin{equation}
\rho=kmr,
\end{equation}
the radial equation can be written
\begin{equation}
\rho^2 \frac{d^2 Z}{d \rho}
+\rho \frac{d Z}{d \rho}+
\left \lbrack \rho^2- \left ( m+\frac{1}{2} \right ) \right \rbrack Z=0.
\end{equation}
This is Bessel's equation, and the solutions are Bessel functions of
the first,  second, and third  kind. These are also well documented in
\textsection 9 and \textsection 10 of Abramowitz and Stegun 
\cite{AbramowitzStegun:Mybib}.
The Bessel functions of the first and second kinds are usually denoted by
$J_\nu$ and $Y_\nu$ respectively. In our case $\nu=m+1/2$ where $m$ is an integer. Some authors refer to the Bessel functions of the second kind as Neumann functions, and denote them as $N_\nu$. Bessel functions of the third kind are also
called Hankel functions, and these can be written as $H_\nu=J_\nu+i Y_\nu$.

Now, if we have constants $c_m$ and $d_m$, the solution of the radial
equation is
\begin{equation}
R=c_m j_m (kmr)+d_m y_n(kmr),
\end{equation}
where the spherical Bessel functions are related to Bessel functions
of fraction order via
\begin{equation}
j_m(\rho)=\sqrt{\frac{\pi}{2 \rho}} J_{(m+1/2)}(\rho)
$$   $$
y_m(\rho)=\sqrt{\frac{\pi}{2 \rho}} Y_{(m+1/2)}(\rho)
\end{equation}
This is sometimes written as
\begin{equation}
r R=c_m \psi_m(kmr)+d_m \chi(kmr),
\end{equation}
where
\begin{equation}
\psi_m(\rho)=\sqrt{\frac{\pi \rho }{2 }} J_{(m+1/2)}(\rho)
$$   $$
\chi_m(\rho)=\sqrt{\frac{\pi \rho }{2 }} Y_{(m+1/2)}(\rho).
\end{equation}
The $\psi$ and $\chi$  are the Riccati Bessel functions.
We note that if $c_m$ is real and $d_m=i c_m$, then we have solutions in terms of Hankel functions.

So, we may now finally write down the solution to the scalar Helmholtz equation in spherical coordinates as
\begin{equation}
u=\sum_{n=0}^{n=\infty}
\sum_{l=0}^{l=m}
P_m^l(\cos \theta) [ c_m j_m(kmr)+d_m y_m(kmr) ] 
[a_l \cos (l \phi) +b_l \sin(l \phi).
\end{equation}
Again, this is sometimes written as
\begin{equation}
r u=\sum_{n=0}^{n=\infty}
\sum_{l=0}^{l=m}
P_m^l(\cos \theta) [ c_m \psi_m(kmr)+d_m \chi_m(kmr) ] 
[a_l \cos (l \phi) +b_l \sin(l \phi)].
\end{equation}

\subsection{The Vector Equation in Spherical Coordinates}
Now, what we really need is a solution to eqn.41. So, we require the Laplacian of a vector field in spherical coordinates. The complication here is that the usual unit vectors for spherical geometry depend on the coordinates. If we write them in terms of rectangular Cartesian coordinates with unit vectors ${\bf i}, \>
 {\bf j}$,  and $\bf k$, then the unit vectors for the spherical system are,
\begin{equation}
{\bf e}_r= \sin \theta \cos \phi {\bf i}
+ sin \theta \sin \phi {\bf j} +\cos \theta {\bf k},
$$    $$
{\bf e}_\theta= \cos  \theta \cos \phi {\bf i}
+ \cos  \theta \sin \phi {\bf j} -\sin  \theta {\bf k},
$$   $$
{\bf e}_\phi= -\sin \phi {\bf i}
+  \cos \phi {\bf j}. 
\end{equation}
We shall also use 
\begin{equation}
{ \bf n}= \cos \phi {\bf i}
+  \sin \phi {\bf j}. 
\end{equation}
So, for instance, let's look at
\begin{equation}
\nabla^2 {\bf e}_r
=
\left \lbrack \frac{1}{r^2 \sin \theta} \frac{\partial }{\partial \theta}
\left ( sin \theta \frac{\partial }\partial \theta \right )
+\frac{1}{r^2 \sin^2 \theta} \frac{\partial^2 } {\partial \phi^2}
\right \rbrack 
( \sin \theta [\cos \phi {\bf i}
  \sin \phi {\bf j} ] +\cos \theta {\bf k} ). 
\end{equation}
This is  
\begin{equation}
\nabla^2 {\bf e}_r=\frac{1}{r^2 \sin \theta}
 \frac{\partial }{\partial \theta} 
\left ( \frac{1}{2} \sin 2 \theta (\cos \phi {\bf i}+\sin \phi {\bf j} )
 -\sin^2 \theta {\bf k} \right )
-\frac{1}{r^2 \sin \theta }( \cos \phi {\bf i}+\sin \phi {\bf j} )
$$    $$
=\frac{1}{r^2 \sin \theta}
\left ( (1 - 2 \sin^2 \theta) (\cos \phi {\bf i}+\sin \phi {\bf j} )
-2 \sin \theta \cos \theta {\bf k} \right )
-\frac{1}{r^2 \sin \theta }( \cos \phi { \bf i}+\sin \phi {\bf j} )
$$   $$
=-\frac{2}{r^2} {\bf e_r}.
\end{equation}
We note that none of our basis vectors ${\bf e}$ are functions of $r$, and 
that
\begin{equation}
\frac{1}{r^2 \sin \theta} \frac{\partial }{\partial \theta}
\sin \theta \frac{\partial f }{\partial \theta}
=\frac{\cos \theta}{r^2 \sin \theta}
\frac{\partial f }{\partial \theta}+\frac{1}{r^2}
\frac{\partial^2 f} {\partial \theta^2}.
\end{equation}

Now, we may write any vector field as following.
\begin{equation}
{\bf A}=A_r(r, \theta, \phi){\bf e}_r
+A_\theta(r, \theta, \phi){\bf e}_\theta+A_\phi(r, \theta, \phi)
 {\bf e}_\phi.
\end{equation}
We introduce the following operators
\begin{equation}
L_r=\frac{1}{r^2} \frac{\partial}{\partial r} 
 r^2 \frac{\partial}{\partial r}, 
\end{equation}
\begin{equation}
L_\theta=
\frac{1}{r^2 \sin \theta} \frac{\partial }{\partial \theta}
\sin \theta \frac{\partial f }{\partial \theta}, 
\end{equation}
and
\begin{equation}
L_\phi=\frac{1}{r^2 \sin^2 \theta}
\frac{\partial^2}{\partial \phi^2}.
\end{equation}
We need the product rules for these operators, which are
\begin{equation}
L_r[u(r)v(r)]
= u L_r v  +v L_r u + 
2 \frac{\partial u}{\partial r}
 \frac{\partial v}{\partial r},
\end{equation}
\begin{equation}
L_\theta[u(\theta)v(\theta)]
= u L_\theta v +v L_\theta u + 
\frac{2}{r^2} \frac{\partial u}{\partial \theta}
 \frac{\partial v}{\partial \theta},
\end{equation}
and
\begin{equation}
L_\phi [u(\phi)v(\phi)]
\theta
= u L_\phi v +v L_\phi u + 
\frac{2}{r^2 \sin^2} \frac{\partial u}{\partial \phi}
 \frac{\partial v}{\partial \phi}.
\end{equation}

What we want is 
\begin{equation}
[L_r+L_\theta+L_\phi][A_r(r, \theta, \phi){\bf e}_r
+A_\theta(r, \theta, \phi){\bf e}_\theta+A_\phi(r, \theta, \phi)]
 {\bf e}_\phi.
\end{equation}
so we must use the product rules above. Alongside these we note that
\begin{equation}
\cos \theta \> {\bf e_r}-\sin \theta \> {\bf e}_\theta
={\bf k}, 
$$ and   $$
\sin  \theta \> {\bf e_r}+\cos  \theta \> {\bf e}_\theta
={\bf n}.
\end{equation}
Using all these, we eventually arrive at
\begin{equation}
\nabla^2 {\bf A}
={\bf e}_r \left \lbrack
\nabla^2 A_r -\frac{2 A_r}{r^2}
- \frac{ 2 A_\theta \cos \theta}{r^2 \sin \theta}
- \frac{2}{r^2} \frac{\partial A_\theta}{\partial \theta}
-\frac{2}{r^2 \sin \theta} \frac{\partial A_\phi}{\partial \phi}
\right \rbrack
$$    $$
+ {\bf e_\theta} \left \lbrack
\nabla^2 A_\theta + 
\frac{2}{r^2} \frac{\partial A_r}{\partial \theta}
-\frac{A_\theta}{r^2 \sin^2 \theta}
-\frac{2 \cos \theta}{r^2 \sin^2 \theta} 
\frac{\partial A_\phi}{\partial \phi} \right \rbrack
$$   $$
+{\bf e_\phi} \left \lbrack
\nabla^2 A_\phi
+ \frac{2}{r^2 \sin^2 \theta} \frac{\partial A_r}{\partial \phi}
+ \frac{2 \cos \theta}{r^2 \sin^2 \theta}
\frac{\partial A_\theta}{\partial \phi}
-\frac{A_\phi}{r^2 \sin^2 \theta} \right \rbrack.
\end{equation}

Now, we shall attempt to start putting together a solution. If one
 imagines a spherical wave, the oscillations (in the far field) will be
perpendicular to the direction of the wave. That is to say the component
  $A_r$ in the ${\bf e}_r$ direction is zero. Looking at this 
component only, we have
\begin{equation}
\left \lbrack
- \frac{ 2 A_\theta \cos \theta}{r^2 \sin \theta}
- \frac{2}{r^2} \frac{\partial A_\theta}{\partial \theta}
-\frac{2}{r^2 \sin \theta} \frac{\partial A_\phi}{\partial \phi}
\right \rbrack=0, 
\end{equation}
or
\begin{equation}
 \frac{  A_\theta \cos \theta}{ \sin \theta}
+ \frac{\partial A_\theta}{\partial \theta}
+\frac{1}{\sin \theta} \frac{\partial A_\phi}{\partial \phi}=0.
\end{equation}
At this point, it is a matter of "playing around". After some consideration
we see that $A_\theta$ and $A_\phi$ might be related to the derivatives
of some function $G$ w.r.t $\theta$ and $\phi$. Indeed, if we put a
trial solution
\begin{equation}
A_\theta=F_1(\theta)\frac{\partial G}{\partial \phi}
$$     $$
A_\phi=F_2(\theta)\frac{\partial G}{\partial \theta}.
\end{equation}
It is soon found that
\begin{equation}
\left \lbrack
\frac{\cos \theta}{\sin \theta} F_1
+ \frac{d F_1}{d \theta} 
\right \rbrack
\frac{\partial G}{\partial \phi}
+\left \lbrack
F_1 + \frac{F_2}{\sin \theta} 
\right \rbrack
\frac{\partial^2 G}{\partial \theta \partial \phi}=0.
\end{equation}
Now it is clear that of we put
\begin{equation}
F_1=\frac{1}{\sin \theta}, F_2=-1,
\end{equation}
 we have 
\begin{equation}
A_\theta=\frac{1}{\sin \theta} \frac{\partial G}{\partial \phi}, \>
A_\phi=-\frac{\partial G}{\partial \theta}.
\end{equation}

As a brief aside, we write down the gradient, divergence, and the curl
operators in spherical coordinates.
\begin{equation}
\nabla f=
\frac{\partial f}{\partial r}{\bf e}_r
+ \frac{1}{r} \frac{\partial f}{\partial \theta}{\bf e}_\theta
+ \frac{1}{r \sin \theta} \frac{\partial f}{\partial \phi}
{\bf e}_\phi,
$$    $$
\nabla \cdot {\bf A}
=\frac{1}{r^2} \frac{\partial }{\partial r} (r^2 A_r)
+\frac{1}{r \sin \theta} \frac{\partial }{\partial \theta}
( \sin \theta A_\theta )
+\frac{1}{r \sin \theta} \frac{\partial A_\phi}{\partial \phi},
$$   and $$
\nabla \times {\bf A}=
{\bf e}_r \frac{1}{r \sin \theta} \left \lbrack
\frac{\partial}{\partial \theta} (A_\phi \sin \theta)
-\frac{\partial A_\theta}{\partial \phi}
\right \rbrack
+{\bf e}_\theta \frac{1}{r}  \left \lbrack
 \frac{1}{ \sin \theta}\frac{\partial A_r}{\partial \phi} 
-\frac{\partial}{\partial r} ( r A_\phi)
\right \rbrack
$$    $$
+{\bf e}_\phi \frac{1}{r}  \left \lbrack
 \frac{\partial }{\partial r} (r A_\theta)
-\frac{\partial A_r}{\partial \theta}. 
\right \rbrack.
\end{equation}
Now suppose we take the curl of some scalar function times
${\bf e}_r$. We write this field as ${\bf M}$, so
\begin{equation}
{\bf M}=\nabla \times [u(r, \theta, \phi) ]{\bf e}_r,
\end{equation}
(This ${\bf M}$  has nothing to do with the magnetisation field.)
If {\bf c} is a vector, in general
\begin{equation}
\nabla \times u {\bf c}=u \nabla \times {\bf c}
+ \nabla u \times {\bf c}.
\end{equation}
We see immediately that $ \nabla \times {\bf e}_r=0$, so that 
\begin{equation}
{\bf M}=\left \lbrack
{\bf e}_r \frac{\partial }{\partial r}
+{\bf e}_\theta \frac{1}{r}\frac{\partial}{\partial \theta}
+{\bf e}_\phi \frac{1}{r \sin \theta}\frac{\partial}{\partial \phi}
\right \rbrack \times [u(r, \theta \phi )] {\bf e}_r
$$    $$
= \nabla \times u {\bf e}_r=\nabla u \times{\bf e}_r
={\bf e}_\theta \frac{1}{r \sin \theta}
\frac{\partial u}{\partial \phi}-
 {\bf e}_\phi \frac{\partial u}{\partial \theta}.
\end{equation}
So, we have components of a field given by eqn.83, but 
divided by $r$. What we have is a solution with no radial component,
 and any solution may be formed by taking the curl of some scalar
 field times ${\bf e}_r$.

This is an important result, so we restate it. If the vector Laplacian
of a field has no radial component, then that field can be represented
 by the curl of a scalar field times ${\bf e}_r$. So, now we can use
some standard results of vector calculus, which apply in any 
coordinate system. For instance, it immediately follows that
\begin{equation}
\nabla \cdot {\bf M}=0.
\end{equation}
From this, it follows that we can put
\begin{equation}
\nabla \times \nabla=-\nabla^2+ \nabla \nabla \cdot=-\nabla^2
$$ and $$
\nabla \times \nabla \times \nabla=-\nabla^2 \nabla \times
=-\nabla \times \nabla^2.
\end{equation}
This is provided of course that the divergence of the field is zero.

It is tempting to use eqn.85, and write
\begin{equation}
\nabla^2 {\bf M}+k^2 m^2 {\bf M}
=\nabla \times \nabla \times \nabla \times u {\bf e}_r
$$    $$
=\nabla^2 \nabla \times u {\bf e}_r+k^2 m^2 \nabla \times u {\bf e}_r
  \nabla \times \nabla^2 u{\bf e}_r+k^2 m^2 \nabla \times u
{\bf e}_r,    
\end{equation}
and then put
\begin{equation}
\nabla^2 {\bf M}+k^2 m^2 {\bf M}=\nabla \times
[\nabla^2 (u {\bf e_r})+k^2 m^2 u {\bf e}_r].
\end{equation}
However, that would be a major blunder! The swapping of the
order has a hidden assumption, and that is that the divergence
of $u {\bf e}_r$ is zero, which it is not.
We must put
\begin{equation}
\nabla \times \nabla \times \nabla \times=
\nabla \times ( \nabla (\nabla \cdot))-\nabla^2 
\end{equation}
instead.
Using the standard results for vector fields, with $A_r=u$,
 $A_\theta=A_\phi=0$, it is relatively easy to  find the gradient of the divergence and the Laplacian terms. Remembering that $u$ is divided by $r$, and
that
\begin{equation}
 \nabla^2 \frac{1}{r}= 4 \pi \delta (r), 
\end{equation}
where the $\delta$ denotes a Dirac delta function,
one arrives at
\begin{equation}
\nabla^2 {\bf M}+k^2 m^2 {\bf M}=
\nabla \times [\nabla^2 u+k^2 m^2 u]{\bf e}_r.
\end{equation}
Now we have our result,
 {\it if $u$ is a solution of the scalar equation, then the curl of 
u ${\bf e}_r$ is a solution of the vector equation.}

But of course, we only have a solution with no radial component.
One might expect that the curl of this solution may well be useful.
Indeed, one can put
\begin{equation}
{\bf N}=\frac{\nabla \times {\bf M}}{k m},
\end{equation}
and then take
\begin{equation}
\nabla^2 {\bf N}=\frac{1}{k m} \nabla^2 \nabla \times {\bf M}
= \frac{1}{k m} \nabla \times \nabla^2 {\bf M}
$$     $$
=-k m \nabla \times {\bf M}=-k^2 m^2 {\bf N}.
\end{equation}
Similarly $\nabla \times {\bf N} = k m {\bf M}.$
So, we now have two independent fields denoted ${\bf M}$ and ${\bf N}$, and
the general solution to the Helmholtz equations will be a linear combination of the two.

Now, we have chosen to use the function $u {\bf e}_r$ as the {\it generating function} as it is called. We could easily have chosen $\psi {\bf r}$. In fact, this choice is more usual. So, we shall now write our general solution as
\begin{equation}
r \psi =\sum_{n=0}^{n=\infty}
\sum_{l=0}^{l=m}
P_m^l(\cos \theta) [ c_m \psi_m(kmr)+d_m \chi_m(kmr) ] 
[a_l \cos (l \phi) +b_l \sin(l \phi)].
\end{equation}

\section{The scattering of a Plane Wave From a Sphere}

Now we have the general form of the solution for the Helmholtz equation in spherical coordinates, we may proceed to use a plane wave as the incoming radiation, and applied boundary conditions so we can actually determine the unknown coefficients in the expansion.

Considering eqns. 35 and 37, and the relations between ${\bf M}$ and ${\bf N}$
 we have
\begin{equation}
{\bf E}={\bf M}_v + i {\bf N}_u
$$       $$ 
{\bf H}= m ( -{\bf M}_u+{\bf N}_v).
\end{equation}
Here the subscripts $u$ and $v$ have the meaning that the generating functions
are $u$ and $v$.

Now,  we have an incident plane wave, this will be written in terms of an infinite series in the same form as the general solution. It turns out, that there is a formula due to Bauer that enables us to do just this \cite{Watson:Mybib}.

Our incident plane wave is to be written down first.
\begin{equation}
E^i_r=e^{-ikr \cos \theta} \sin \theta \cos \phi
$$    $$
E^i_\theta=e^{-ikr \cos \theta} \sin \theta \sin \phi
$$    $$
E^i_\phi=e^{-ikr \cos \theta}  \sin \phi,
\end{equation}
and 
\begin{equation}
H^i_r=e^{-ikr \cos \theta} \sin \theta \cos \phi
$$    $$
H^i_\theta=e^{-ikr \cos \theta} \sin \theta \sin \phi
$$    $$
H^i_\phi=e^{-ikr \cos \theta}  \cos \phi.
\end{equation}
The wave propagates in the $z$ direction, with the electric field being plane polarised in the $x$ direction.  The superscript $i$ serves to denote that this is the incident wave.

The formula due to Bauer is
\begin{equation}
e^{ikr \cos \theta}=\sum_{n=0}^{n=\infty} (-i)^n (2n+1) 
\frac{\psi_n(k r)}{kr} P_n(\cos \theta),
\end{equation}
where
\begin{equation}
\psi_n(\rho)= \sqrt{ \frac{\pi \rho}{2}} J_{n+1/2}(\rho).
\end{equation}
Also needed are
\begin{equation}
e^{i kr \cos \theta} \sin \theta
=\frac{1}{ikr} \frac{\partial}{\partial \theta} e^{i kr \cos \theta},
\end{equation}
and
\begin{equation}
\frac{\partial}{\partial \theta} P_n(\cos \theta)=-P^1_n(\cos \theta).
\end{equation}
After some work, we have the generating functions $u^i$ and $v^i$
 for the incident field, 
\begin{equation}
ru^i= \frac{1}{k} \sum_{n=0}^{n=\infty} (-i)^n
\frac{ 2n+1}{n(n+1)} \psi_n (k r) P^1_n(\cos \theta) \cos \phi
$$    $$
rv^i= \frac{1}{k} \sum_{n=0}^{n=\infty} (-i)^n
\frac{ 2n+1}{n(n+1)} \psi_n (k r) P^1_n(\cos \theta) \sin \phi.
\end{equation}
Of course, any functions which diverge  in the domain where they are applied 
must have coefficients set to zero.

For the scattered wave
\begin{equation}
ru^s= \frac{1}{k} \sum_{n=0}^{n=\infty} (-i)^n
\frac{ 2n+1}{n(n+1)} a_n \xi_n (k r) P^1_n(\cos \theta) \cos \phi,
$$    $$
rv^s= \frac{1}{k} \sum_{n=0}^{n=\infty} (-i)^n
\frac{ 2n+1}{n(n+1)} b_n \xi_n (k r) P^1_n(\cos \theta) \sin \phi.
\end{equation}
Here, the $\xi$  are formed from the spherical Hankel functions in the same 
way that $\psi$ and $\chi$ were formed from the spherical Bessel functions
of the first and second kind.  The  Hankel functions do not diverge at infinity. and for the internal wave (with superscript $t$), we must use functions that do not diverge at the origin. That is
\begin{equation}
ru^t= \frac{1}{k} \sum_{n=0}^{n=\infty} (-i)^n
\frac{ 2n+1}{n(n+1)} c_n \psi_n (k r) P^1_n(\cos \theta) \cos \phi,
$$    $$
rv^t= \frac{1}{k} \sum_{n=0}^{n=\infty} (-i)^n
\frac{ 2n+1}{n(n+1)} d_n \psi_n (k r) P^1_n(\cos \theta) \sin \phi.
\end{equation}
Given a sphere of radius $a$ we must match
\begin{equation}
E^i_\theta +E^i_\theta=E^t_\theta
$$ and   $$
E^i_\phi +E^i_\phi=E^t_\phi,
\end{equation}

and the other usual boundary conditions at an interface (see \textsection 33 of \cite{FeynLect:Mybib}. This leads to
\begin{equation}
u^u+u^s=mu^t, \> \> v^i+v^s=v^t,
\end{equation}
and 
\begin{equation}
\frac{\partial}{\partial r}[r (u^i+u^s)]
 =\frac{1}{m} \frac{\partial}{\partial r} (r u^t)
\frac{\partial}{\partial r}[r (v^i+v^s)]
 =\frac{\partial}{\partial r} (r v^t).
\end{equation}
Putting in the series expansions above, we arrive at
\begin{equation}
m[\psi^\prime_n(ka)-a_n \xi^\prime_n(ka)]=c_n \psi^\prime(kma)
$$   $$
m[\psi^\prime_n(ka)-b_n \xi^\prime_n(ka)]=d_n \psi^\prime(kma)
$$   $$
[\psi_n(ka)-a_n \xi_n(ka)]=c_n \psi(kma)
$$   $$
[\psi_n(ka)-b_n \xi_n(ka)]=d_n \psi(kma).
\end{equation}
Eliminating the $c_n$ and $d_n$, and putting $x=ka$ and $y=mx$ leaves us with
\begin{equation}
a_n=\frac{\psi^\prime_n(y)\psi_n(x)-m \psi_n(y) \psi^\prime_n(x)}{
   \psi^\prime_n(y) \xi_n(x)-m \psi_n(y) \xi^\prime_n(x)}.
\end{equation}
\begin{equation}
b_n=\frac{m \psi^\prime_n(y)\psi_n(x)- \psi_n(y) \psi^\prime_n(x)}{
  m \psi^\prime_n(y) \xi_n(x)- \psi_n(y) \xi^\prime_n(x)}.
\end{equation}
Similarly,
\begin{equation}
c_n=\frac{ m[\psi^\prime_n(x)\xi_n(x)- \psi_n(x) \xi^\prime_n(x)]}{
   \psi^\prime_n(y) \xi_n(x)-m \psi_n(y) \xi^\prime_n(x)}.
\end{equation}

\begin{equation}
d_n=\frac{ m[\psi^\prime_n(x)\xi_n(x)- \psi_n(x) \xi^\prime_n(x)]}{
   m \psi^\prime_n(y) \xi_n(x)- \psi_n(y) \xi^\prime_n(x)}.
\end{equation}

In general, we shall only be interested in the field outside the sphere, so only the $a_n$ and $b_n$ are required. In particular, we shall be interested in the far field, in which case we can use the approximation
\begin{equation}
\xi_n(kr) \approx i^n e^{-ikr}.
\end{equation}
Then
\begin{equation}
r u^s=-\frac{ i e^{-ikr} \cos \phi}{k}
\sum_{n=0}^\infty 
\frac{2n+1}{n(n+1)} a_n
P^1_n(\cos \theta),
\end{equation}
\begin{equation}
r v^s=-\frac{ i e^{-ikr} \sin \phi}{k}
\sum_{n=0}^\infty 
\frac{2n+1}{n(n+1)} b_n
P^1_n(\cos \theta).
\end{equation}
In the far field all the radial components vanish.
Then, from eqn.87 and eqn.98
\begin{equation}
E^s_\theta=-\frac{i}{kr} e^{-ikr} \cos \phi \sum_{n=0}^\infty
\frac{2n+1}{n(n+1)}
\left \lbrack a_n \frac{d}{d \theta} P^1_n( \cos \theta)
 +b_n \frac{P^1_n(\cos \theta)}{ \sin \theta} \right \rbrack,
\end{equation}
and
\begin{equation}
E^s_\phi=-\frac{i}{kr} e^{-ikr} \sin \phi \sum_{n=0}^\infty
\frac{2n+1}{n(n+1)}
\left \lbrack a_n \frac{P^1_n( \cos \theta)}{\sin \theta} P^1_n( \cos \theta)
 +b_n \frac{ d}{ d \theta} P^1_n(\cos \theta) \right \rbrack.
\end{equation}
It is customary to put $\cos \theta=\mu$, and it should be noted that $P^1_0(\mu)=0.$ 

Now, {\it in the plane of scattering}, $\phi$ is fixed by definition. We expect that there will be some component $E^s_l$ that is parallel to the plane of the plane of scattering, and some component $E^s_r$ that is perpendicular to it. In this case $\phi$ is eliminated, and the two fields are functions of $\theta$ alone.

\section{The Scattering Functions and the Phase Matrix}

It is standard to define two scattering functions as follows
\begin{equation}
k A_1=S_1=\sum_{n=1}^{\infty} \frac{2n+1}{n(n+1)} (a_n \pi_n(\mu) 
+b_n \tau_n{\mu}),
\end{equation}
and
\begin{equation}
k A_2=S_2=\sum_{n=1}^{\infty} \frac{2n+1}{n(n+1)} (b_n \pi_n(\mu) 
+a_n \tau_n{\mu}).
\end{equation}
Here we have introduced the angular functions $\pi_n$ and $\tau_n$.
They are 
\begin{equation}
\pi_n(\mu)=\frac{P^1_n (\mu)}{\sin \theta}, \> \> \tau_n(\mu)=\frac{d}{d \theta} P^1_n(\mu).
\end{equation}
There are a few important relations that allow us to calculate these functions easily. These are
\begin{equation}
\pi_n=\frac{d}{d \mu} P_n(\mu),
\end{equation}
\begin{equation}
\tau_n(\mu)=\mu \pi_n(\mu)-(1-\mu^2) \frac{d}{d \mu} \pi_n(\mu).
\end{equation}
From the theory of Legendre Polynomials \cite{AbramowitzStegun:Mybib} we have
\begin{equation}
P_n(\mu)=\frac{1}{2^n n!} \frac{d^n}{d \mu^n} (\mu^2-1)^n.
\end{equation}
This allows us to establish recurrence relations
\begin{equation}
\pi_n=\cos \theta  \> \frac{2n-1}{n-1}\pi_{n-1}- \frac{n}{n-1} \pi_{n-2}
$$   $$
\tau_n= \cos \theta \> [ \pi_n-\pi_{n-2}]-(2n-1) \sin^2 \theta \pi_{n-1}+\tau_{n-2}.
\end{equation}
In practice we need start up values, these are
\begin{equation}
\pi_0=0, \> \> \tau_0=0,
$$    $$
\pi_1=1, \> \> \tau_1=\cos \theta,
$$ and $$
\pi_2(\theta)=3 \cos \theta, \> \> \tau_2( \theta)= 3 \cos 2 \theta.
\end{equation}
For the forward and back scattering directions $(\theta=0, \> \theta=\pi)$ we have
\begin{equation}
\pi_n(0)=\tau_n(0)=\frac{n (n+1)}{2},
$$ and $$
-\pi_n(\pi)=\tau_n(\pi)=(-1)^n \frac{n (n+1)}{2}.
\end{equation}

Now we return to eqn.121 and eqn.122 which define the scattering functions.
In matrix form, we write these as
\begin{equation}
\pmatrix{ E^s_l \cr E^s_r}= \pmatrix{ S_1 & 0 \cr 0 & S_2 } 
\pmatrix{ E^i_l \cr E^s_r}.
\end{equation}
We will be interested in the scattered energy rather than the fields themselves,that is 
\begin{equation}
i_1=k^2 A_1 A^*_1= S_1 S^*_1,
$$   $$
i_2=k^2 A_2 A^*_2= S_2 S^*_2,
\end{equation}
 where as usual the asterisk denotes the complex conjugate.

Now, there may well be phase differences between the parallel and perpendicular fields, and this will lead to the possibility of elliptical or circular polarisation. In which case we must describe the incident and scattered intensities in terms of the Stokes' parameters (see for instance \textsection 15 of Chandrasekhar's Radiative Transfer \cite{Chandrasekhar:Mybib}. A section on the Stokes's parameters is also given in Deirmendjian \cite{Deirmendjian:Mybib} or any standard optics textbook such as Born and Wolf's Principles of Optics \cite{BornWolf:Mybib}.

When cast in these terms we have
\begin{equation}
\pmatrix{ I^s_l \cr I^s_r \cr U^s \cr V^s }
= \pmatrix{
    A_1 A^*_1 & 0 & 0 & 0 \cr
    0 & A_2 A^*_2 & 0 & 0 \cr
    0 & 0 & Re(A_1 A^*_2) & -Im( A_1 A^*_2) \cr
    0 & 0 & Im(A_1 A^*_2) & Re( A_1 A^*_2) 
}
\pmatrix{ I^i_l \cr I^i_r \cr U^i \cr V^i }
\end{equation}
The Poynting Vector ${\bf P}$ describes energy fluxes, 
\begin{equation}
 {\bf P}=\frac{1}{2} R_e( {\bf E} \times {\bf H}).
\end{equation}
From this the scattering cross section can be calculated.  It turns out that
\begin{equation}
\sigma_{sc}=\frac{1}{2} \int_\Omega (A_1 A^*_1 + A_2 A^*_2) d \omega.
\end{equation}
Here, $\Omega$ represents the integral over $4 \pi$ steradians, and $d \omega$ is a differential of solid angle. This gives us another important quantity, the scattering efficiency $K_{sc}$. This turns out to be
\begin{equation}
K_{sc}=\frac{2}{x^2} \sum_{n=1}^\infty (2n+1)(a_n a^*_n +b_n b^*_n).
\end{equation}
The extinction efficiency includes absorption and is given by
\begin{equation}
K_{ex}=\frac{2}{x^2} \sum_{n=1}^\infty (2n+1) Re(a_n+b_n).
\end{equation}

The ratio of the two is of particular interest. This is the single
scattering albedo denoted $\varpi_0$
\begin{equation}
\varpi_0=\frac{K_{sc}}{K_{ex}}.
\end{equation}
If nothing that interacts with the sphere is scattered, then everything 
that interacts with the sphere is absorbed: the single scattering albedo is zero. If everything that interacts with the sphere scatters, then extinction and scattering efficiencies are the same and the single scattering albedo is one.


Also of interest in some applications is the radar or exact backscattering cross section
\begin{equation}
\sigma_b=\frac{ 4 \pi}{k^2} \vert S_1(\pi) \vert^2,
\end{equation}
with
\begin{equation}
-S_1(\pi)=S_2(\pi)=\frac{1}{2}\sum_{n=1}^\infty (-1)^n (2n+1) (a_n-b_n).
\end{equation} 
This leads also to the definition of the backscattering efficiency or normalised radar cross section,
\begin{equation}
K_b=\frac{ \sigma_b}{\pi a^2}i.
\end{equation}

At this stage we can introduce the phase function and the phase matrix. This will finally round off the discussion of the scattering of light from a single sphere. First we note that we can define a normalisation condition. This is simply saying that, given a scattering efficiency, the integral over a sphere of all the scattered radiation gives all the incident radiation that is scattered.
That is
\begin{equation}
\frac{1}{2 \pi} \int_\omega \left ( \frac{2 i_1(\theta)}{x^2 K_{sc}}
 + \frac{2 i_2(\theta)}{x^2 K_{sc}} \right ) \> d \omega=1.
\end{equation}

This leads naturally to the definition of four functions,
\begin{equation}
P_j(\theta)=\frac{ 4 i_j(\theta)}{x^2 K_{sc}(x)}
=\frac{ 4 \sigma_j(\theta)}{a^2 K_{sc}(x)}, \> \> j=1,2,3,4.
\end{equation}
And in these terms the scattering rule
\begin{equation}
\pmatrix{ I^s_l \cr I^s_r \cr U^s \cr V^s }
=  \frac{\pi a^2 K_{sc} }{ 4\pi} \pmatrix{
    P_1 & 0 & 0 & 0 \cr
    0 & P_2 & 0 & 0 \cr
    0 & 0 & P_3  & P_4 \cr
    0 & 0 & -P_4 &  P_3
}
\pmatrix{ I^i_l \cr I^i_r \cr U^i \cr V^i }.
\end{equation}
For many applications, only {\it the scalar phase function} $P(\theta) $ is needed, where
\begin{equation}
P(\theta)=\frac{1}{2}[P_1(\theta)+P_2(\theta)].
\end{equation}
To finish off this section, yet another important quantity is called the asymmetry parameter and is denoted $g$.
\begin{equation}
g=\int_0^{\pi} \cos \theta P(\theta) d\theta.
\end{equation}
This gives the ratio of how much energy is scattered in the forward hemisphere directions, to the energy backscattered in the other $2 \pi$ steradians.
In terms of the coefficients $a_n$ and $b_n$, the asymmetry parameter is given by
\begin{equation}
g= \frac{4}{ x^2 K_{sc}} \sum_{n=1}^{N} \left \lbrack
  \frac{n(n+2}{n+1} Re( a_n a_{n+1}^* +b_n b_{n+1}^*)
    + \frac{2n+1}{n(n+1)} Re(a_n b_n^*) \right \rbrack
$$   $$
g= \frac{4}{ x^2 K_{sc}} \sum_{n=1}^{N} \left \lbrack
  \frac{(n+1)(n-1)}{n} Re( a_{n-1} a^* +b_{n-1} b_n^*)
    + \frac{2n+1}{n(n+1)} Re(a_n b_n^*) \right \rbrack
\end{equation}
Note that this relies on a finite expansion, so that $a_{N+1}=b_{N+1}$ is defined as zero, which 
gives rise to the second expression.

\section{Afterword}
We have reviewed much of the (basic) theory of the scattering of electromagnetic radiation from a 
sphere, but further material (and corrections) may be added at some later stage. The author 
notes that any potential reader may need to calculate the scattering coefficients, 
phase function and so on for whatever research purpose. 

The authors general advice is not to go off and write a programme. The author 
advises the use of Warren Wiscombe's programme MieV0 
which is  available at
\newline
 ftp://climate1.gsfc.nasa.gov/wiscombe  
(along with much other useful programmes and data). 
This includes the report NCAR/TN-140+STR
edited and revised in 1996. (NCAR being the National Centre for Atmospheric Research). 
The last formulae for the asymmetry parameter were taken from that report.
The author also points to his other article Mie2 at
 http://seagods.stormpages.com 
where numerical details are looked into, and the integration over size distributions of particles.
This also points to downloadable computer code which calculates the phase function 
directly in terms of the Mie $a$ and $b$ coefficients without any need for the angular functions
$\pi_n$ and $\tau_n$.
\bibliography{../../Mybib}

\end{document}

